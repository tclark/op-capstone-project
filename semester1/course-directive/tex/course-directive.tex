\documentclass{article}
\usepackage{graphicx}
\usepackage{wrapfig}
\usepackage{ucs}
\usepackage[utf8x]{inputenc}
\usepackage{enumerate}
\usepackage{hyperref}
\usepackage[margin = 2.25cm]{geometry}




\begin{document}

\begin{figure}
%\includegraphics[width=30mm]{../../../resources/images/oplogo.png}
\end{figure}

\title{Course Directive\\IN700001 Capstone Project \\Semester One, 2016}
\date{}
\maketitle

\section*{Description}
In this paper our aim is to carry out advanced project work in the information technology field, applying skills learned in the degree programme. We will demonstrate commitment, competence, creativity, and craft throughout the process. In particular we will initiate the development of large, real, client-driven projects that we will complete in the second semester.

Throughout the semester we will also take time to reflect on what we are doing, why we are doing it, and how we can improve.

\section*{Course Information}
\begin{itemize}
  \item 15 Credits
  \item Prerequisite: IN602 Software Engineering
\end{itemize}

\section*{Lecturers}
\begin{tabular}{lr}

  % after \\: \hline or \cline{col1-col2} \cline{col3-col4} ...
  Tom Clark &    \\
     Office: & D205b \\
     Phone: & 03 470 4356 \\
     Email: & \texttt{tom.clark@op.ac.nz} \\
     GitHub: & \url{https://github.com/tclark} 
\end{tabular}

\smallskip

\noindent Elvis Adomnic\unichar{259} and David Rozado will also serve in project advisory roles this semester.

\section*{Course Dates}
\begin{tabular}{ll}
Term 1:  & 15 February - 8 April\\
Term 2:  & 25 April - 17 June\\
\end{tabular}



\section*{Learning Outcomes}
On completion of this paper you will be able to:
\begin{enumerate}
  \item Carry out project work to produce a functional, deployed system for the client;
  \item Review progress and develop a plan for the final robust delivery of the full year project.
 
\end{enumerate}

\section*{Course Content}
Our overall goal this semester - and continuing into the next one - is to build, operate, and maintain a functioning information technology product or service. Everything we do will contribute to or follow from this goal.  Every team member is expected to contribute meaningfully to every facet of the project, including coding, testing, documenting, deploying, and maintaining it. You will have opportunities to apply and hone your strengths, but you will also improve in areas where you are weaker.

We will manage our work using the \emph{Scrum} development method over four time boxed sprints this semester. After each sprint we will review our work and look for ways to improve our process and output in the next sprint. In this way we expect to extend our products to provide more desired features over time, while at the same time improving our own abilities to deliver them. We will also follow a \emph{continuous delivery} model. While we will explore this more thoroughly over this year, basically this means that we will always maintain our code in a release-ready state, and we will deploy our projects to a production platform by the end of every sprint.

\section*{Resources}

Our primary working spaces are D206 and D205. D204 is our dedicated hardware workshop. D206 is fitted out so that each student in Project 1 will have a dedicated workspace with a desk and a computer. This computer will have the standard BIT lab image installed at the beginning of the semester, but it is assumed that this will not be sufficient for most project work. You have both the freedom to install whatever you need for your project work and the corresponding responsibility to maintain your own workstation. Some support for this is available from Rob Broadley, the BIT sysadmin, as well as other staff, but the responsibilty is your own. You must register for after hours access at \url{http://afterhours.ict.op.ac.nz} so that you may use your swipe card to access these spaces throughout the year.

Note that these are shared working areas and all project participants are required to use and maintain the spaces in a way that allows all project participant to use the space safely and productively. This will be considered part of your project performance and it may factor into your mark.

There is a budget for necessary project resources, primarily hardware.  All purchases must be approved by the lecturer.

Project code and related artifacts will be hosted on GitHub, so all students must have accounts.  Our intention is that these items will be useful in your job search, so keep that in mind when establishing your profile.


	\subsection*{Proprietary Software}
	Much of your project work will be done with Free/Open Source Software (FOSS) and this preferred over proprietary software when possible. We will, however, be pragmatic in our approach and it is likely that you will be required to use some proprietary software to complete this paper.

\section*{Schedule}
We have one scheduled classroom session each week on Tuesdays from 8:00 to 10:00 AM. We will not always meet at that time, but you should be available to meet at that time each week. In addition to this meeting time, each project team will have a variety of scheduled meetings and working sessions during the week that you will be expected to attend. You will also have individual meetings with project lecturers from time to time.
 

Our approximate schedule for the semester is as follows, subject to change based on the needs of the class and the demands of particular projects.

\medskip

\renewcommand{\arraystretch}{1.5}
\begin{tabular}{|l|c|l|}
\hline
 Week & Week Start & Topic                                \\ \hline
 1    & 15 Feb     & Software Engineering Review          \\ \hline
 2    & 22 Feb     & Software Engineering Tools           \\ \hline
 3    & 29 Feb     & DevOps Practices                     \\ \hline
 4    &  7 Mar     & Project Introductions and Assignment \\ \hline
 5    & 14 Mar     & Research                             \\ \hline
 6    & 21 Mar     & Sprint 1                             \\ \hline
 7    & 28 Mar     & Sprint 1                             \\ \hline
 8    &  4 Apr     & Sprint Review                        \\ \hline
 H1   & 11 Apr     & Holiday                              \\ \hline
 H2   & 18 Apr     & Holiday                              \\ \hline
 9    & 25 Apr     & Sprint 2                             \\ \hline
 10   &  2 May     & Sprint Review                        \\ \hline
 11   &  9 May     & Sprint 3                             \\ \hline
 12   & 16 May     & Sprint 3                             \\ \hline
 13   & 23 May     & Sprint Review                        \\ \hline
 14   & 30 May     & Sprint 4                             \\ \hline
 15   &  6 Jun     & Sprint 4                             \\ \hline
 16   & 13 Jun     & Sprint Review                        \\ \hline
\end{tabular}



\section*{Assessment}
Strictly speaking, this paper has only one assessment: your project itself. We will, however, evaluate your project work with two measures this semester. First, we will evaluate the overall quality of your project. We will then scale your mark according to your relative contributions to the team's work. For example, if a project receives an overall mark of 80, but one team member has been judged to have done only three-fourths of a share fo the work, then that student will receive only three-fourths of the overall mark.

At about two times during the semester, each student will receive individual professional development objectives and will be marked on the degree to which they are met. These items will comprise 20\% of the semester's mark.

\medskip


\begin{tabular}{|l|c|}
\hline
Assessment                          & Weighting \\ \hline
Individual Professional Development &  20\% \\ \hline
Project Work Performance            &  80\% \\ \hline
\end{tabular}


\section*{Criteria for Passing}
You must earn an overall average mark of 50\% or better to pass this paper.

\newpage

\section*{Course Requirements and Expectations}
\subsection*{Attendance}
\begin{itemize}
 \item Students are expected to attend all classes, meetings, and working sessions.
 \item If you must miss a scheduled activity, you must make arrangements with team members and other afftected parties.
\end{itemize}

\subsection*{Communication}
Important announcements and discussions about the course, assessments, and scheduling may take place during class sessions and meetings.  It is your responsibility to be informed about them.  If you cannot attend a session, be sure to check with another student.

This year we will use Slack as a group communication tool. Sign up at http://op-bit.slack.com and make a habit of monitoring relevant channels.

Your student email is an official communication channel. It is your responsibility to regularly check your student email for important course related material, including changes to class scheduling or assessment details. Not checking will not be accepted as an excuse.

You can manage your email at the Student Hub and download the instructions for forwarding your email at \url{http://www.op.ac.nz/students/student-hub/}

\subsection*{Polytechnic Closure}
In the event that the Polytechnic is closed or has a delayed opening because of snow or bad weather you should not attempt to come to campus if it is unsafe to do so. It is possible that your instructor will not be able to attend either, so sessions may not physically meet. However, this does not become a holiday. Rather, work can be performed online and from your home to the fullest extent possible. Information about closure will be posted on the Otago Polytechnic Facebook page \url{https://www.facebook.com/OtagoPoly} and status updates will be posted on Slack.

\subsection*{Group Work and Originality}
Students in the Bachelor of Information Technology degree are expected to hand in original work.  This project is somewhat unusual in that all work is deeply collaborative and you are required to discuss and share your work with others. The essential thing is that you must not claim another's work as your own, and doing so will be regarded as plagarism.

\subsection*{Referencing}
Appropriate referencing is required for all work.  Referencing standards will be specified by your instructor.

\subsection*{Plagiarism}
Plagiarism is submitting someone else's work as your own.  Plagiarism offences are taken seriously and an
assessment that has been plagiarised may be awarded a zero mark.  A definition of plagiarism is in the Student Handbook,
available online or at the school office.

\subsection*{Submission Requirements}
All work is to be submitted by the time, date, and method given when it is issued.

\subsection*{Extensions}
Extensions are only available for unusual circumstances.  These must be applied for, and approved, prior to the submission deadline.

\subsection*{Impairment}
In case of sickness contact your lecturer or year co-ordinator as soon as possible.  The policy regarding the granting of a mark that considers impaired performance requires a medical certificate and a medical practitioners signature on a form. You may should refer to the guide on impaired performanceon the student handbook.

\subsection*{Appeals}
If you are concerned about any aspect of your assessment, please approach the lecturer in the first instance.  We support
an open door policy and aim to resolve issues promptly.  Further support is available from the Programme
Manager and Head of School. Otago Polytechnic has a formal process for academic appeals if necessary.

\subsection*{Other Documents}
Regulatory documents relating this course can be found on the Polytechnic website.

\subsection*{Special Resources and Requirements}
If you have any special needs, whether they relate to the course material, the exercises, the assessment, or anything in the course -
then \textit{please} let your instructor know as soon as possible.

\end{document}
